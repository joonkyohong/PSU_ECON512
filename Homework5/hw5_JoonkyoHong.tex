\documentclass[10pt, letterpaper]{article}

\usepackage{amssymb,amsmath,amsfonts,eurosym,geometry,ulem,graphicx,caption,color,setspace,sectsty,comment,footmisc,caption,natbib,pdflscape,subfigure,array,hyperref}
\usepackage{xcolor}
\usepackage{mathptmx}
\usepackage{listings}
\lstset{language=Matlab}
\hypersetup{
    colorlinks,
    linkcolor={red!50!black},
    citecolor={blue!50!black},
    urlcolor={blue!80!black}
}
\normalem
\usepackage[flushleft]{threeparttable}
    
\onehalfspacing
\newtheorem{theorem}{Theorem}
\newtheorem{corollary}[theorem]{Corollary}
\newtheorem{proposition}{Proposition}
\newenvironment{proof}[1][Proof]{\noindent\textbf{#1.} }{\ \rule{0.5em}{0.5em}}

\newtheorem{hyp}{Hypothesis}
\newtheorem{subhyp}{Hypothesis}[hyp]
\newtheorem{asu}{Assumption}
\renewcommand{\thesubhyp}{\thehyp\alph{subhyp}}

\newcommand{\red}[1]{{\color{red} #1}}
\newcommand{\blue}[1]{{\color{blue} #1}}

\newcolumntype{L}[1]{>{\raggedright\let\newline\\arraybackslash\hspace{0pt}}m{#1}}
\newcolumntype{C}[1]{>{\centering\let\newline\\arraybackslash\hspace{0pt}}m{#1}}
\newcolumntype{R}[1]{>{\raggedleft\let\newline\\arraybackslash\hspace{0pt}}m{#1}}

\geometry{left=1.0in,right=1.0in,top=1.0in,bottom=1.0in}

\begin{document}

\title{Homework 5: ECON512}
\author{Joonkyo Hong}
\date{}
\maketitle
\smallskip

\noindent For question 1 and 2, the relevant values are -1257.07 and -796.589, respectively.

\noindent Tables below are summarizing starting values of the functions, initial parameters, maximized values of log-likelihood function, and the arg max for each case. For this exercise, rather than estimating $\sigma_{\beta,u}$, I estimate $\rho$ which is the correlation between $\beta_{i}$ and $u_{i}$. For the first two cases, I employ logit estimates without constant for the initial values for $\gamma$ and $\beta_{0}$ and one for $\sigma_{\beta}$. For the last case, I employ logit estimates with constant and the constant is the initial value for $u_{0}$. For $\sigma_{u}$, I use one.

\begin{table}[h!]
  \begin{center}
    \caption{Simple Model with Gaussian Quadrature}  
    \label{tab:start}
    \begin{tabular}{c|cccccc|c} % <-- Alignments: 1st column left, 2nd middle and 3rd right, with vertical lines in between
      \hline\hline
       Case        &$ \gamma  $&$  \beta_{0} $&$ \sigma_{\beta} $& function value\\
      \hline
       Initial     & -0.3272  &  1.4935      &      1            &    -571.7613 \\
       At maximum  & -0.5056  &  2.4832      &    1.4055         &    -536.2378 \\    
      \hline      \hline
    \end{tabular}
  \end{center}
\end{table} 

\begin{table}[h!]
  \begin{center}
    \caption{Simple Model with quasi Monte Carlo}  
    \label{tab:start}
    \begin{tabular}{c|cccccc|c} % <-- Alignments: 1st column left, 2nd middle and 3rd right, with vertical lines in between
      \hline\hline
                     &$ \gamma  $&$  \beta_{0} $&$ \sigma_{\beta} $& function value\\
      \hline
       Initial     & -0.3272  &  1.4935        &    1            &    -111.1193 \\       
       At Maximum  & -0.5056  &  2.5579        &    1.3962       &    -76.0706  \\       
      \hline      \hline
    \end{tabular}
  \end{center}
\end{table} 


\begin{table}[h!]
  \begin{center}
    \caption{Full Model with quasi Monte Carlo}  
    \label{tab:start}
    \begin{tabular}{c|cccccc|c} % <-- Alignments: 1st column left, 2nd middle and 3rd right, with vertical lines in between
      \hline\hline
       Case            &$ \gamma  $&$  \beta_{0} $&$   u_{0}  $&$ \sigma_{\beta} $&$ \sigma_{u}  $&$ \rho $& function value\\
      \hline   
       Initial         & -0.3613  &  1.4825      &  0.7384    &    1            &     1         &    0     &    7.4112    \\ 
       At Maximum      & -0.6759  &  2.9606      &  1.3015    &    1.8491       &  1.5358       &  0.4311  &   65.6452    \\ 
      \hline      \hline
    \end{tabular}
  \end{center}
\end{table} 


\clearpage

\begin{verbatim}
%%%%%%%%%%%%%%%%%%%%%%%%%%%%%%%%%%%%%%%%%%%%%%%%%%%%%%%%%%%
% Homework #5 ECON 512                                    %
% Written by Joonkyo (Jay) Hong, 18 Nov 2018              %
%%%%%%%%%%%%%%%%%%%%%%%%%%%%%%%%%%%%%%%%%%%%%%%%%%%%%%%%%%%
clear;

addpath('./CEtools/');   % First, add path CEtools %

% Load dataset

load 'hw5.mat';

x=reshape(data.X,100*20,1);
z=reshape(data.Z,100*20,1);
y=reshape(data.Y,100*20,1);
dta = [y x z];

% Option set for fmincon
options_opt = optimoptions('fmincon', 'Algorithm', 'interior-point', 'Display', 'iter', 'StepTolerance',1e-30,...
      'TolFun',1e-4,'MaxIter',1e2, 'MaxFunEvals', 1e8);
  
% Initial Values for Q1 and Q2 (Logit Estimates)
init_theta = glmfit([z x],y,'binomial','link','logit','constant','off');
init_theta = [init_theta' 1];

% Initial Values for Q4 (Logit Estimates)
init_theta2 = glmfit([z x],y,'binomial','link','logit');
init_theta2 = [init_theta2(2:3,1)' init_theta2(1,1) 1 1 0];


%% Question 1. Simple with Gaussian Quadrature Approach

    k = 20;                                     % # of nodes for GQ approach
    lik_fcn1 = @(theta) loglik_GQ(theta,dta,k);

  % Constraints in parameter space
  % gamma, beta0, sigma_beta

   lb = [-inf -inf   0];    
   ub = [inf   inf inf];

  % Initial Values for Optimization and Initial value of likelihood function
  
   fcnval_for_Q1 = -lik_fcn1([0, 0.1, 1]);
   init_fcn1 = -lik_fcn1(init_theta);
   
  % Minimizing minus log likelihood

   tic
   [theta_mle1,fval1,exitflag1,~] = fmincon(lik_fcn1,init_theta,[],[],[],[],lb,ub,[],options_opt);
   time1 = toc;

%% Question 2. Simple with Quasi-Monte Carlo

    k = 100;                                     % # of nodes for q-MC approach
    lik_fcn2 = @(theta) loglik_qMC(theta,dta,k);

  % Constraints in parameter space
  % gamma, beta0, sigma_beta

   lb = [-inf -inf   0];    
   ub = [inf   inf inf];

  % Initial Values for Optimization and Initial value of likelihood function
  
   fcnval_for_Q2 = -lik_fcn2([0, 0.1, 1]);   
   init_fcn2 = -lik_fcn2(init_theta);
   
  % Minimizing minus log likelihood

   tic
   [theta_mle2,fval2,exitflag2,~] = fmincon(lik_fcn2,init_theta,[],[],[],[],lb,ub,[],options_opt);
   time2 = toc;


%% Question 4. Full with Quasi-Monte Carlo

    k = 200;                                     % # of nodes for q-MC approach
    lik_fcn4 = @(theta) loglik_full(theta,dta,k);

  % Constraints in parameter space
  % gamma, beta0, u0, sigma_beta, sigma_u, rho

   lb = [-inf -inf -inf   0   0 -1];    
   ub = [inf   inf  inf inf inf  1];

  % Initial Values for Optimization and Initial value of likelihood function
  
   init_fcn4 = -lik_fcn4(init_theta2);
   
  % Minimizing minus log likelihood

   tic
   [theta_mle4,fval4,exitflag4,~] = fmincon(lik_fcn4,init_theta2,[],[],[],[],lb,ub,[],options_opt);
   time4 = toc;


%% Question 5. Report the results


 disp("              ");
 disp("Case 1: Simple with Gaussian Quadrature");
 disp("Initial");
 disp(num2str(init_theta));
 disp("MLE Estimates");
 disp(num2str(theta_mle1));
 disp("Function Value");
 disp("Initial");
 disp(num2str(init_fcn1));
 disp("Maximand");
 disp(num2str(-1*fval1));


 disp("              ");
 disp("Case 2: Simple with quasi-Monte Carlo");
 disp("Initial");
 disp(num2str(init_theta));
 disp("MLE Estimates");
 disp(num2str(theta_mle2));
 disp("Function Value");
 disp("Initial");
 disp(num2str(init_fcn2));
 disp("Maximand");
 disp(num2str(-1*fval2));

 
 disp("              ");
 disp("Case 3: Full with quasi-Monte Carlo");
 disp("Initial");
 disp(num2str(init_theta2));
 disp("MLE Estimates");
 disp(num2str(theta_mle4));
 disp("Function Value");
 disp("Initial");
 disp(num2str(init_fcn4));
 disp("Maximand");
 disp(num2str(-1*fval4));

\end{verbatim}
                     

\end{document}