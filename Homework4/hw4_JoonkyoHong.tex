\documentclass[10pt, letterpaper]{article}

\usepackage{amssymb,amsmath,amsfonts,eurosym,geometry,ulem,graphicx,caption,color,setspace,sectsty,comment,footmisc,caption,natbib,pdflscape,subfigure,array,hyperref}
\usepackage{xcolor}
\usepackage{mathptmx}
\usepackage{listings}
\lstset{language=Matlab}
\hypersetup{
    colorlinks,
    linkcolor={red!50!black},
    citecolor={blue!50!black},
    urlcolor={blue!80!black}
}
\normalem
\usepackage[flushleft]{threeparttable}
    
\onehalfspacing
\newtheorem{theorem}{Theorem}
\newtheorem{corollary}[theorem]{Corollary}
\newtheorem{proposition}{Proposition}
\newenvironment{proof}[1][Proof]{\noindent\textbf{#1.} }{\ \rule{0.5em}{0.5em}}

\newtheorem{hyp}{Hypothesis}
\newtheorem{subhyp}{Hypothesis}[hyp]
\newtheorem{asu}{Assumption}
\renewcommand{\thesubhyp}{\thehyp\alph{subhyp}}

\newcommand{\red}[1]{{\color{red} #1}}
\newcommand{\blue}[1]{{\color{blue} #1}}

\newcolumntype{L}[1]{>{\raggedright\let\newline\\arraybackslash\hspace{0pt}}m{#1}}
\newcolumntype{C}[1]{>{\centering\let\newline\\arraybackslash\hspace{0pt}}m{#1}}
\newcolumntype{R}[1]{>{\raggedleft\let\newline\\arraybackslash\hspace{0pt}}m{#1}}

\geometry{left=1.0in,right=1.0in,top=1.0in,bottom=1.0in}

\begin{document}

\title{Homework 4: ECON512}
\author{Joonkyo Hong}
\date{}
\maketitle
\smallskip

\noindent For question 1 to 4, I use 100 draws (or nodes) to approximate $\pi$.

~\\

\noindent Q1. (Quasi Monte Carlo with Dart Throwing: q-MC with DT)

\begin{enumerate}
\item Draw $\{ x_{i}, y_{j} \}_{i=1 ~ j=1}^{100 ~ 100}$ from \texttt{qnwequi}
 over $[0,1] \times [0,1]$.
\item Compute 
   \begin{equation}
    z_{ij}
    \begin{cases}
      1, & \text{if}\ x_{i}^{2} + y_{j}^{2} \leq 1 \nonumber \\
      0, & \text{otherwise}
    \end{cases}
  \end{equation} 
\item Approximate $\frac{\pi}{4}$ as $\frac{\#~\text{of}~z_{ij}=1}{100 \times 100}$
\end{enumerate}


\noindent Q2. (Newton-Cotes with Dart Throwing: NC with DT)

\begin{enumerate}
\item Create grids $\{ x_{i}, y_{j} \}_{i=0 ~ j=0}^{100 ~ 100}$ where $x_{i} = \frac{i}{100}$ and $y_{j} = \frac{j}{100}$.
\item Create the matrix $Z$ whose $(i,j)$-th element is 
   \begin{equation}
    z_{ij}
    \begin{cases}
      1, & \text{if}\ x_{i}^{2} + y_{j}^{2} \leq 1 \nonumber \\
      0, & \text{otherwise}
    \end{cases}
  \end{equation} 
\item Following Simpson, create the 101 by 1 weight vector $w$ as
   \begin{align}
    &  w_{0}=w_{101} = \frac{0.01}{3} \nonumber \\    
    &  w_{k} = 
    \begin{cases}
      \frac{4}{3} \times 0.01, & \text{if $k$ is even} \nonumber \\
      \frac{2}{3} \times 0.01, & \text{if $k$ is odd}
    \end{cases}
  \end{align} 
\item Approximate $\frac{\pi}{4}$ as $w'Zw$
\end{enumerate}



\noindent Q3. (Quasi Monte Carlo with Pythagorean formula: q-MC with PYT)

\begin{enumerate}
\item Draw $\{ x_{i}\}_{i=1}^{100}$ from \texttt{qnwequi}
\item Approximate $\frac{\pi}{4}$ as $\frac{1}{100}\sum_{i=1}^{100}\sqrt{1-x_{i}^{2}}$
\end{enumerate}

\newpage

\noindent Q4. (Newton-Cotes with Pythagorean formula: NC with PYT)

\begin{enumerate}
\item Create grids $\{ x_{i} \}_{i=0 ~ j=0}^{100}$ where $x_{i} = \frac{i}{100}$.
\item Create the vector $f$ whose $i$-th element is 
   \begin{equation}
    f_{i}=\sqrt{1-x_{i}^{2}} \nonumber
  \end{equation} 
\item Following Simpson, create the 101 by 1 weight vector $w$ as
   \begin{align}
    &  w_{0}=w_{101} = \frac{0.01}{3} \nonumber \\    
    &  w_{k} = 
    \begin{cases}
      \frac{4}{3} \times 0.01, & \text{if $k$ is even} \nonumber \\
      \frac{2}{3} \times 0.01, & \text{if $k$ is odd}
    \end{cases}
  \end{align} 
\item Approximate $\frac{\pi}{4}$ as $w'f$
\end{enumerate}

Here are the approximated $\pi$'s and the relevant absolute errors to the true $\pi$.
\begin{table}[h!]
  \begin{center}
    \caption{Approximated $\pi$}  
    \label{tab:estimate}
    \begin{tabular}{c|c|c} % <-- Alignments: 1st column left, 2nd middle and 3rd right, with vertical lines in between
      \hline\hline
       Method                & Approx. $\pi$     &   Abs. Error    \\
      \hline
       q-MC with DT          & 3.1513           &  0.0304          \\
       NC with DT            & 3.1425           &  0.0009         \\ 
       q-MC with PYT         & 3.1387           &  0.0104          \\
       NC with PYT           & 3.1416           &  0.0005          \\ 
      \hline      \hline
    \end{tabular}
  \end{center}
\end{table} 




\noindent Q5. 

\noindent To conduct pseudo Monte Carlo, I draw uniform random variables at each simulation.  

Here are the Mean Squared errors for each methods with draws (grids) 100, 1,000, and 10,000, respectively (For Newton-Cotes, the reported values are squared error).

\begin{table}[h!]
  \begin{center}
    \caption{Approximated $\pi$}  
    \label{tab:estimate}
    \begin{tabular}{c|c|c|c} % <-- Alignments: 1st column left, 2nd middle and 3rd right, with vertical lines in between
      \hline\hline
       Method                & N=100           &  N=1,000       &  N=10,000  \\
      \hline
       pseudo MC with DT     & 0.0015           &  0.0002       &  0.0001     \\
       NC with DT            & 8.849e-07        &  1.127e-09    &  8.551e-13    \\ 
       pseudo MC with PYT    & 0.0079           &  0.0007       &  7.081e-05  \\
       NC with PYT           & 2.111e-07        &  2.109e-10    &  2.109e-13 \\ 
      \hline      \hline
    \end{tabular}
  \end{center}
\end{table} 

Note that with small number of draws, pseudo Monte Carlo methods are showing poor performance in approximating $\pi$. In particular, the pseudo-MC with Dart Throwing approach is the worst way to compute $\pi$. Overall, Newton-Cotes with Pythagorean approach outperforms all the other methods in terms of squared errors. 

\clearpage

\begin{verbatim}
%%%%%%%%%%%%%%%%%%%%%%%%%%%%%%%%%%%%%%%%%%%%%%%%%%%%%%%%%%%
% Homework #4 ECON 512                                    %
% Written by Joonkyo (Jay) Hong, 20 Oct 2018              %
% Modified on 22 Oct 2018                                 %
%%%%%%%%%%%%%%%%%%%%%%%%%%%%%%%%%%%%%%%%%%%%%%%%%%%%%%%%%%%
clear;

addpath('./CEtools/');   % First, add path CEtools %
N = 100;                 % # of draws or nodes 

%% Questions 1 (Dart-throwing method with quasi-MC approach)

     [x1, ~]= qnwequi(N, [0 0], [1 1]);
     z   =  indic_fcn(x1(:,1),x1(:,2));
     
     pi1 =  4*mean(mean(z));
     error1 = abs(pi1 - pi);
     
%% Question 2 (Dart-throwing method with Newton-Coates approach)
 
     pi2 = 4*Int_indic([0 0],[1 1],N,N);     
     error2 = abs(pi2 - pi);

%% Questions 3 (Pythagorean method with quasi-MC approach)

     [x3, ~]  = qnwequi(N,0,1);
     pi3 = 4*mean(sqrt(1-x3.^2));
     error3 = abs(pi3 - pi);
     
%% Questions 4 (Pythagorean method with Newton-Coates approach)

     pi4 = 4*Int_simp(@(x) sqrt(1-x.^2), 0, 1, N);
     error4 = abs(pi4 - pi);
     
     
 result_from_1_to_4 =  ...
  [" ",           "approximate pi", "absolute error";
   "q-MC with DT",     pi1,               error1      ;
   "NC with DT"  ,     pi2,               error2      ;
   "q-MC with PYT",    pi3,               error3      ;
   "NC with PYT" ,     pi4,               error4      ];
%% Question 5 (Pseudo-MC)

N       = [100; 1000; 10000]; 
num_sim = 200;                   % # of simulations
store_array = zeros(2,length(N),num_sim);  % 3D-array that will store the Squared errors


for n=1:length(N)   
            
    for i=1:num_sim
        
          % Dart-Throwing with Pseudo-MC
        
          x1 = rand(N(n),2);    % Pseudo-MC 
         
          z   =  indic_fcn(x1(:,1),x1(:,2));
          pi1 =  4*mean(mean(z));
          store_array(1,n,i) = (pi1 - pi)^2;        
                    
          % Pythagorean with Pseudo-MC
        
          x3 = rand(N(n),1);    % Pseudo-MC 
         
          pi3 = 4*mean(sqrt(1-x3.^2));
          store_array(3,n,i) = (pi3 - pi)^2;
          
    end
    
          % Dart-Throwing with Newton-Cotes
           
          pi2 = 4*Int_indic([0 0],[1 1],N(n),N(n));     
          store_array(2,n,:) = (pi2 - pi)^2;
          
          % Pythagorean with Newton-Cotes

          pi4 = 4*Int_simp(@(x) sqrt(1-x.^2), 0, 1, N(n));
          store_array(4,n,:) = (pi4-pi)^2;
          
          
end



                   
results_mat = mean(store_array,3);   % Calculate mean over the simulations
results_mat= [[" ", "N=100", "N=1,000", "N=10,000"];
 ["pseudo-MC with DT"; "NC with DT"; "pseudo-MC with PYT"; "q-MC with PYT"], (results_mat)];

disp(" ");
disp(" ");
disp("Question 1~4. Approximated pi and abs error: N=100");
disp(result_from_1_to_4);

disp(" ");
disp("Question 5. Mean Squared Errors");
disp(results_mat);

\end{verbatim}
                     

\end{document}